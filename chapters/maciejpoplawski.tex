\section{Maciej Poplawski}
\begin{math}
\left| G \right| = \left| G : H \right| * \left| H \right|
\\
\left| z \right| = \sqrt{a^2 + b^2}
\end{math} \\
Twierdzenie Lagrange'a i modul liczby zespolonej

\begin{figure}[h]
    \centering
    \includegraphics[width=0.5\textwidth]{pictures/driver.jpg} 
    \caption{Robert Kubica (Funkcja odwrotna)}
    \label{fig:obrazek}
\end{figure}

Oto przykladowa lista numerowana:

\begin{enumerate}
    \item Pierwszy element
    \item Drugi element
    \item Trzeci element
\end{enumerate}

A tutaj przykladowa lista nienumerowana:

\begin{itemize}
    \item Jablko
    \item Banan
    \item Gruszka
\end{itemize}

\begin{table}[]
\begin{tabular}{lll}
\textbf{Jezyk programowania} & \textbf{Data wydania} & \textbf{Tworca}            \\
C++                          & 1985                  & Bjarne Stroustrup          \\
Python                       & 1991                  & Python Software Foundation \\
Java                         & 1996                  & James Gosling             
\end{tabular}
\label{tab:words}
    \caption{Jezyki programowania}
\end{table}

\section{Akapity}

\textsc{Zebry} to zwierzeta roślinozerne, które zamieszkuja przede wszystkim sawanny i stepy Afryki. Sa latwo rozpoznawalne dzieki swojemu charakterystycznemu umaszczeniu, skladajacemu sie z \textbf{czarno-bialych} pasków. Te paski nie tylko nadaja im unikalny wyglad, ale pelnia równiez funkcje \textit{kamuflazu}, pomagajac zmylić drapiezniki, takie jak lwy, poprzez rozbijanie ich sylwetki w stadzie. Ciekawostka jest, ze kazdy wzór pasków na ciele zebry jest niepowtarzalny, podobnie jak odciski palców u ludzi. Dzieki temu mozna odróznić poszczególne osobniki w stadzie.

\textsc{Zebry} sa zwierzetami spolecznymi, zyjacymi w grupach zwanych stadami, ktore zapewniaja im wieksza ochrone przed drapieznikami. Ich dieta sklada sie glownie z traw, a ze wzgledu na specyfike trawienia, zebry musza spedzac znaczna czesc dnia na zerowaniu. Mimo ze sa roslinozerne, ich uzebienie jest przystosowane do gryzienia twardych roslin i skutecznego przyswajania pozywienia bogatego w \underline{blonnik}. W czasie migracji moga pokonywac duze odleglosci w poszukiwaniu swiezych pastwisk, co czyni je wytrwalymi i odpornymi na trudne warunki srodowiskowe zwierzetami.